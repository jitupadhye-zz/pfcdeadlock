\secspacelarge
\section{Deadlock in Lossless Network}
\secspace

We now briefly discuss how deadlocks form in lossless networks, and why we must
study the necessary and sufficient conditions of deadlock formation.

%\para{Lossless Ethernet relies on PFC.} Modern datacenter networks require
%efficient network stacks in order to support high bandwidth and low latency.
%RDMA over Ethernet, or RoCE, is one of the most popular solutions and has been
%deployed in Google and Microsoft datacenters~\cite{dcqcn, timely}.  The key
%property of RoCE is that it runs on a lossless Ethernet, in which no packet can
%be dropped due to congestion.  Lossless is achieved by enabling PFC
%(Priority-based Flow Control). With the PFC PAUSE mechanism, a switch can pause
%an incoming link when its ingress buffer occupancy reaches a preset threshold.
%Properly tuned, the buffer is never full, and no packet is dropped due to
%insufficient buffer space.  Unfortunately, deadlock may occur in such lossless
%networks.

\para{Lossless Ethernet relies on PFC.} \revise{RoCE needs PFC to provide a lossless L2 network. With the PFC PAUSE mechanism, a switch can pause
an incoming link when its ingress buffer occupancy reaches a preset threshold.
Properly tuned, no packet will be dropped due to
insufficient buffer space.  Unfortunately, deadlock may occur in such lossless
networks.}

\para{PFC may lead to deadlock, if paused links form a cycle.} \revise{In a PFC-enabled network, if a subset of links simultaneously paused by PFC happen to form a directed cycle, no packets in the cycle can move even if there is no more new traffic injected into this cycle.}

%This essentially means that there is {\em cyclic buffer dependency} -- every
%link is waiting for its next hop to RESUME the link, and since there is a cycle,
%no link makes progress.

To avoid such deadlock, {\em deadlock-free routing}~\cite{tcpbolt} has been
proposed. It guarantees that (if the routing configuration is correct,) any
traffic does not cause deadlock.

\para{Unfortunately, achieving deadlock-free routing is inefficient, and may not
even be viable.} Deadlock-free routing is achieved by eliminating Cyclic Buffer
Dependency (CBD)~\cite{deadlockfree}.  However, ensuring that there is never any
CBD is challenging.

First, deadlock-free routing largely limits the choice of topology. \revise{For example,
Stephens et al. \cite{tcpbolt} proposes to only use tree-based topology and routing, and shows
that it is deadlock-free.  However, there are a number of
other datacenter topologies and routing schemes that are not
tree-based~\cite{bcube, camcube, jellyfish}, and do not have deadlock-free
guarantee.}

Second, due to bugs or misconfiguration, deadlock-free routing configuration may
turn into deadlock-vulnerable. In fact, recent work has observed a PFC deadlock
case in real-world tree-based datacenter\cite{rdmascale}, caused by the
(unexpected) flooding of lossless class traffic. \revise{This case is a concrete example to show that even for tree-based topology, CBD can still occur if up-down routing is not strictly followed.} Furthermore, there are multiple
reports of routing loops due to misconfiguration in today's production
datacenters~\cite{everflow, libra}. If lossless traffic encounters any of these
loops, CBD is unavoidable.

In this paper, we argue that we should accept the fact that CBD
cannot be completely avoided,\footnote{In other words, deadlock-free
routing may not always apply or work correctly.} and instead try to
understand more precise deadlock conditions.  Our findings show that even if
there is CBD, deadlock may not occur (see
Section~\ref{sec:analysis}).  This means that CBD is only a
necessary but not sufficient condition for deadlock.  We show the occurrence of
deadlock is affected by the packet TTL, the traffic matrix, as well as flow
rate. Based on these findings, we propose several ways to avoid deadlock even in
face of CBD.


