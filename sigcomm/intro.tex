%\vspace{-0.1in}
\section{Introduction}\label{sec:intro}

\para{Datacenter appliations demand high performance from the network. RDMA provides such performance.
We deploy RDMA over Converged Ethernet (RoCE).} \\ 

\para{RoCE relies on PFC. PFC can cause deadlocks.} \\

\para{Prior work} proposes theoritical solutions. But none of them is practical for current datacenter deployment.
There are two main issues, 1) most of the solutions require significant changes on routing, enforcing or accurately predicting
the path of each packet. This is conflicted with the nature of Ethernet and IP networks and hard to achieve in practice.
We have evidences from production datacenters that a certain portion of packets are routed unexpectedly, though not dropped.
2) The number of lossless priority available is very limited. There are only several traffic classes in total,
and each lossless class requires siginifant reserved buffer to make PFC work. 

\para{(Our solution)} In this paper, we work on a practical solution that can immediately implemented by 
commodity switches. We do not enforce the path of each packet or make any changes on the routing algorithm. 
This is important for network operators. We also minimize the hardware resources usage, like number of priority
classes, reserved buffer and number of ACL rules.

\para{(Limit the scope)} We do not discuss centralized SDN solutions. BTW, the centralized solution is not trivial, refer
to our hotnets paper. We assume our networks rely on PFC. These solutions would require significant changes 
in the network, and may affect performance.

\para{Specifically,} we seek to design an algorithm that takes the topology and routes that must be lossless as the 
input, and outputs the buffer configuration and ACL rules that guarantee no deadlock. To begin with, we focus on 
common structured datacenter topology, including Fat-tree and BCube. The key idea is to tag the packets along 
the path, and assign each packet to different lossless queues. Properly configured, we can eliminate any
possibilities of deadlock with a small number of lossless queues.

\para{We further extend our algorithms in two directions.} First, we consider generic, unstructured topology. 
We abstract the ideas in designing algorithms for strucutred topology, and develop an algorithm that works for
any graph, using TTL as a generic tag. We show that it works well for random topology like Jellyfish.
Second, in cases where the limit number of lossless queues cannot guarantee deadlock-free, we provide suggestions for 
network operator that which part of routing paths may be turned into lossy so that the whole network can be deadlock-free.

\para{We implement our algorithm and verify it using testbed and NS-3 simulations.} \\

\para{Our contributions.} 1) we analyze the practical challenges in making RoCE deadlock-free, including 
the challenges in enforcing/predicting routing and the limit of lossless queues. 2) we develop practical solutions
that eliminate deadlock for Fat-Tree and BCbue. 3) we extend our algorithm for generic graphs and scenarios
where the number of lossless queues is insufficient.


