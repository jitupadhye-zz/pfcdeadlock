%\vspace{-0.1in}
\section{\sysname{} for Clos Topology}
\label{sec:specific}

While \sysname{} works for any topology, we start by demonstrating the main
ideas on the most popular data center topology, Clos, with common Clos
routing\footnote{Clos routing can be implemented by BGP (Microsoft, Facebook) or
by customized routing protocol (Google).}.  Before delving into the details, we
define the following terms.

\para{Lossless routes:} The input to \sysname{}. It is a set of packet paths
that network operators require to be lossless. For example, in he Clos network,
the operator may say that all up-down paths are lossless.

\para{Lossless class:} A lossless class is a service that we provide for
applications.  Network guarantees that packets in a lossless class will not drop
on lossless routes. 

\para{Tag:} Tag is a small integer assigned / associated with a packet. The tag
is embedded in a packet. A packet belonging to one lossless class may change its
tag value, which means that a lossless class may have multiple tag values. Tags
can be implemented using TTL or special DSCP values.

\para{Switch model:} A switch is composed of multiple ports.  Per port, there
are limited number of separated ingress queues and egress queues.  Switch
classifies packets into queues based on {\em tags}.  Each queue can be lossless
or lossy, depending on whether PFC is enabled for that priority.  All packets
are put into a lossy queue unless there are whitelist rules that classify
specific packets into lossless queues. Lossless ingress queues can trigger PFC,
while lossless egress queues react to PFC from the next hop.

\subsection{Basic idea: Tagging on Bounce}\label{subsec:tag}

\begin{figure}[t]
	%\vspace{-0.1in}
	\centering
	
	\subfloat[x][Basic Clos topology] {
		\includegraphics[width=0.5\textwidth] {figs/cbd_a}
	}

	\subfloat[short for lof][1-bounce paths creates CBD.] {
		\includegraphics[width=0.5\textwidth] {figs/cbd_a}
	}
	
%	\vspace{-0.15in}
	\subfloat[short for lof][CBD is eliminated with path segmenting and prioritizing.]{
		\includegraphics[width=0.5\textwidth] {figs/cbd_b}
	}
	
	\caption{Micropath based priority transition can eliminate CBD.}\label{fig:priority_transition}
\end{figure}

Consider the Clos network in Figure~\ref{fig:priority_transition}(a). Both the
green and blue flows follow up-down routing and there is no deadlock. 

Now, let's assume that as shown in Figure~\ref{fig:priority_transition}(b), two
links fail, which forces both flows to take so-called 1-bounce paths. As shown,
this can lead to CBD, and hence deadlock.

One way to avoid such deadlocks is to put any packets that have deviated from
the up-down routes in a lossy queue -- i.e. assign them a priority value that is
not associated with a lossless class.  Note that being assigned to lossy
priority does not mean that the packets are dropped -- they are dropped only if
they arrive at an ingress queue that is full. In short, these packets will not
trigger PFC. Since the basic up-down paths are deadlock free, and the bounced
packets do not trigger PFC, the network will stay deadlock free even if packets
bounce.

Thus, if each switch can detect that an arriving packet has deviated (sometime
in past) from the up-down path, it can put it in lossy queue, and there will be
no deadlock.

Note that this solution requires no changes to existing routing protocols, and
dynamic routing is handled by assigning packets to lossy queues.

But how can a switch determine whether the packet has bounced, with just local
information?

%% 
%% 
%% Figure~\ref{fig:priority_transition}(a) shows a simple Clos topology. Let's
%% assume that the operator wants all normal paths to be lossless. In addition,
%% let's assume that the operator is also worried about rerouting due to link 
%% failures (\S\ref{sec:reroute}). In this section, we focus on just {\em one}
%% lossless class for applications. We will revisit multiple lossless classes in \S\ref{subsec:reusing}.
%% 
%% Typically, the "normal" paths will be the shortest paths -- which for a CLOS
%% network are the ``up-down'' paths described in the last section.  In addition,
%% due to link failures, we may have paths that are not up-down. We call a hop that
%% is ``down-up'', {\em i.e.,} a packet comes down but then goes up, {\em a bounce}.
%% Such bounced paths can form CBD, and lead to deadlock, as shown in
%% Figure~\ref{fig:priority_transition}(a))~\cite{shpiner2016unlocking}. 
%% 
%% However, if we divide the paths into two segments, {\em before-bounce} and {\em
%% after-bounce}, and assign them to different priority queues, there is no more
%% CBD (Figure~\ref{fig:priority_transition}(b)). 
%% 
%% \para{Key insight.} The insight we get from this
%% example is that if the switch can detect that a packet may cause CBD in next
%% hops, it can change the packet's priority queue to avoid CBD, thus avoiding deadlock.
%% Specifically, in Clos topology, if the switch can detect packet bounce and change 
%% the priority queue, CBD can be eliminated.
%% 
%% Note that this is different from the prior solutions,

Consider the green flow in Figure~\ref{fig:priority_transition}(b).  We
want switch $S1$, the first switch after bounce, to detect the bounce and change
priority queue. However, without additional information, $S1$ cannot
differentiate bounced packets from normal up-going packets, {\em e.g.,} blue
flow packets before bounce. 

It turns out that $L2$ can provide this information.  In normal ``up-down''
routing, $L2$ would never forward a packet that arrived from $S2$ to $S1$. So,
if $L2$ ``tags'' packets that have arrived from $S2$ that it is forwarding to $S1$,
then $S1$, {\em and all other switches along the path after $S1$} know that the
packet has deviated from the up-down path, and put such packets in a lossy queue. 

Note that $L2$ needs only local information to do the tagging. We can initially
assign a $NoBounce$ tag to every packet. $L2$ then needs to check ingress and
egress port: it changes the tag from $NoBounce$ to $Bounced$ if a packet
arriving from ingress port connected to $S2$ exits on egress port connected to
$S1$.  Modern switch hardware indeed provides the functionality to install such
a tagging rule (\S\ref{sec:implemenation}).  It is easy to see that these rules
can be pre-computed since we know the topology, and the set of routes that we
want to be ``lossless'' (i.e. the up-down routes). 

Of course, the basic idea is not enough. First of all, packets may bounce not
just from the spine layer, but at any layer. Also, recall from
\S\ref{sec:reroute} that ``bounces'' are a fact of life in data center networks.
The operator may not want to put packets that have suffered just a single bounce
into a lossy queue -- we may want to wait until the packet has bounced more than
once before assigning it to a lossless queue. And we must ensure that we don't
end up using more lossless queues than the switch can support. We now show how
to do this.

\subsection{Combining Tags}
\label{subsec:combine}

In Clos network as shown in Figure~\ref{fig:priority_transition}(a), packets can
bounce at any $L$ layer, or called Leaf, switches. In addition, packets can also
bounce at any $T$ layer, or ToR switches. Do we need to assign a distinct tag
and a corresponding priority queue for each bouncing point? 

Again, we leverage our knowledge of Clos topology and routing. Consider a packet
that bounces twice. The path between the first bounce and the second bounce is a
normal up-down path. Therefore, these path segments do not have CBD even if we
combine them into a single priority queue. We can use a single tag to represent
these segments altogether, and map the tag to a globally unique priority queue.

This completes the design of \system{} for Clos topology.  We first tag all
packets following the normal ``up-down'' routing input with tag value of 1. All
switches put these packets into first lossless queue. Then we configure all ToR
and Leaf switches such that every time they see a packet coming down and then
going up (therefore, bouncing) for any reasons, they increase the tag by one.
Assuming there are $k$ available lossless queues, all switches put packets with
tag $i$ to $i^{th}$ lossless queues. This means packets with up to $k-1$ bounces
are all lossless. If a packet bounces more than $k-1$ times, it will carry a tag
larger than $k$. All switches will put such packets into a lossy queue.

Note that the design satisfies all three challenges described in
\S\ref{sec:challanges}. We do not change the routing protocol. We work with
existing hardware. We deal with dynamic changes to routes, and we do not exceed
the number of available lossless queues.

We now provide brief proof sketches to prove that the above algorithm is
deadlock free, and optimal in terms of number of lossless priorities used. 

\subsection {Deadlock freedom and optimality}

\begin{figure}[t]
	%\vspace{-0.1in}
	\centering
	
	\subfloat[short for lof][CBD across subspaces.] {
		\includegraphics[width=0.24\textwidth] {figs/subspace_a}
	}
	\subfloat[short for lof][DAG across subspaces.]{
		\includegraphics[width=0.26\textwidth] {figs/subspace_b}
	}
	
	\caption{Ordered subspaces ensure deadlock-free priority transition.}\label{fig:subspace}
\end{figure}

\para{\system{} is deadlock-free for Clos networks:} \system{} has two important
properties. First, for any given tag and its corresponding priority queue, there
is no CBD because each tag only has a set of ``up-down'' routing paths.  Second,
every time the tag changes, it changes monotonically. This means that the packet
is going unidirectionally in a DAG of priority queues. This is important because
otherwise CBD may still happen across subspaces (Figure~\ref{fig:subspace}).  We
conclude that no CBD can form either within a tag or across different tags. The
network is deadlock-free since CBD is a necessary condition of
deadlock~\cite{our_hotnets_paper}.

\para{\system{} is optimal in terms of lossless priorities used:} We show that
to make all $k-1$ bounces paths lossless and deadlock-free, at least $k$
lossless priorities are required. We use contradiction.  Assume there exists a
system with only $k-1$ lossless priorities. Consider a flow that loops between
$T1$ and $S1$ for $k$ times, which means bounces $k-1$ times at $T1$. With
Pigeonhole principle, we know that at least two times during the looping, the
packet must have the same lossless priority. This is already sufficient to cause
a deadlock~\cite{our_hotnets_paper}. Contradiction.

\subsection {Summary}

We showed that given a Clos topology and a set of routes (up-down, plus
upto $k-1$ ``bounces'') that should be lossless, we can generate a system of
tags (and associated rules) that ensure that packets are correctly classified
into $k$ lossless queues to ensure that there is no deadlock.   This system can
be implemented with existing hardware, without requiring any changes to the
routing protocol. We also showed that this system is optimal, in the sense that
we cannot make do with fewer lossless queues.

We now show how to generalize \system{} for any topology.

